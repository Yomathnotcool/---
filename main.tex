\documentclass[a4paper]{article} 
\addtolength{\hoffset}{-2.25cm}
\addtolength{\textwidth}{4.5cm}
\addtolength{\voffset}{-3.25cm}
\addtolength{\textheight}{5cm}
\setlength{\parskip}{0pt}
\setlength{\parindent}{0in}

%----------------------------------------------------------------------------------------
%	PACKAGES AND OTHER DOCUMENT CONFIGURATIONS
%----------------------------------------------------------------------------------------

\usepackage{blindtext} % Package to generate dummy text
\usepackage{charter} % Use the Charter font
\usepackage[utf8]{inputenc} % Use UTF-8 encoding
\usepackage{microtype} % Slightly tweak font spacing for aesthetics
\usepackage[english, ngerman]{babel} % Language hyphenation and typographical rules
\usepackage{amsthm, amsmath, amssymb} % Mathematical typesetting
\usepackage{float} % Improved interface for floating objects
\usepackage[final, colorlinks = true, 
            linkcolor = black, 
            citecolor = black]{hyperref} % For hyperlinks in the PDF
\usepackage{graphicx, multicol} % Enhanced support for graphics
\usepackage{xcolor} % Driver-independent color extensions
\usepackage{marvosym, wasysym} % More symbols
\usepackage{rotating} % Rotation tools
\usepackage{censor} % Facilities for controlling restricted text
\usepackage{listings, style/lstlisting} % Environment for non-formatted code, !uses style file!
\usepackage{pseudocode} % Environment for specifying algorithms in a natural way
\usepackage{style/avm} % Environment for f-structures, !uses style file!
\usepackage{booktabs} % Enhances quality of tables
\usepackage{tikz-qtree} % Easy tree drawing tool
\tikzset{every tree node/.style={align=center,anchor=north},
         level distance=2cm} % Configuration for q-trees
\usepackage{style/btree} % Configuration for b-trees and b+-trees, !uses style file!
\usepackage[backend=biber,style=numeric,
            sorting=nyt]{biblatex} % Complete reimplementation of bibliographic facilities
\addbibresource{ecl.bib}
\usepackage{csquotes} % Context sensitive quotation facilities
\usepackage[yyyymmdd]{datetime} % Uses YEAR-MONTH-DAY format for dates
\renewcommand{\dateseparator}{-} % Sets dateseparator to '-'
\usepackage{fancyhdr} % Headers and footers
\pagestyle{fancy} % All pages have headers and footers
\fancyhead{}\renewcommand{\headrulewidth}{0pt} % Blank out the default header
\fancyfoot[L]{} % Custom footer text
\fancyfoot[C]{} % Custom footer text
\fancyfoot[R]{\thepage} % Custom footer text
\newcommand{\note}[1]{\marginpar{\scriptsize \textcolor{red}{#1}}} % Enables comments in red on margin

%----------------------------------------------------------------------------------------

\usepackage[UTF8]{ctex}
\newtheorem{theorem}{定理}
\newtheorem{example}{例}
\newtheorem{definition}{定义}
\usepackage{tikz-cd}
\begin{document}
%-------------------------------
%	TITLE SECTION
%-------------------------------

\fancyhead[C]{}
\hrule \medskip % Upper rule
\begin{minipage}{0.295\textwidth} 
\raggedright
\footnotesize
姓名:邓志远 \\   
学号170600302 
\end{minipage}
\begin{minipage}{0.4\textwidth} 
\centering 
\large 
数学文化 1\\ 
\normalsize 
范畴下的乘积\\ 
\end{minipage}
\begin{minipage}{0.295\textwidth} 
\raggedleft
\today\hfill\\
\end{minipage}
\medskip\hrule 
\bigskip

%-------------------------------
%	CONTENTS
%-------------------------------

\section{范畴的基本定义}
范畴被称之为数学的数学。在集合论面对Russell悖论时的虚弱无力,除了提出了公理化体系改造传统的集合论,使之成为在满足一族公理化条件deZermelo–Fraenkel集合论。然后更直接的想法是发展一套不以集合为基础的数学语言:范畴论。一言以蔽之,范畴是由一类对象和对象之间的关系所组成的。

\begin{example}
     一个不严谨不是数学的例子:假想全体人类组成的范畴,其他对象之间的关系为社交关系。那么如果两个人的社交关系完全一样的情况下,这两个人在社会意义上就是同一个人了。
\end{example}

\begin{definition}
类(class)是一类有着严格相同性质的集合(set)(或者其他数学对象)。
\end{definition}
\begin{example}
是类但不是集合(set)的例子:
\begin{enumerate}
    \item 所有群组成的全体是类(class),群是一个集合带有满足相应要求运算,但是群的定义中不能确定群中元素;
    \item 所有单点集的全体,所有固定基数的集合全体。
\end{enumerate}
\end{example}
在给出来了类的概念后,可以给出范畴的定义:
\begin{definition}
范畴$\mathfrak{C}$:
\begin{enumerate}
    \item 一个类$\mathbf{Ob}(\mathfrak{C})$其中元素称为范畴中的对象,通常记为大写字母$X,Y,...$,且$X\in\mathfrak{C}$即为$X\in\mathbf{Ob}(\mathfrak{C})$;
    \item 对于任意有序对$X,Y\in\mathfrak{C}$有集合$\mathbf{Hom}(X,Y)$, $\mathbf{Hom}(X,Y)$是对象$X,Y$之家的态射。对于$\phi\in\mathbf{Hom}(X,Y)$,有$\phi:X\rightarrow Y$;
    \item 对于元素之间的态射应该满足一下条件:
    \begin{enumerate}
        \item 态射复合:对于任意三个$X,Y,Z\in\mathfrak{C}$,和其中态射:$\phi:X\rightarrow Y, \psi:Y\rightarrow Z$,则可以得到$X$到$Z$的态射,其称之为$\phi,\psi$的复合态射,记为:$\phi\circ\psi$或者$\phi\psi$;
        \item 复合的结合律:对于$X,Y,Z,U\in\mathfrak{C}$,有态射:$\phi:X\rightarrow Y,\psi:Y\rightarrow Z, \kappa:Z\rightarrow U$,可以得到:$(\kappa\circ\psi)\circ\phi=\kappa\circ(\phi\circ\psi)$;
        \item 态射存在恒等态射:对于每一个$X\in\mathfrak{C}$,存在态射:$1_{X}:X\rightarrow X$,这个态射称为局部恒等态射。对于每一个$Y\in\mathfrak{C}$,且$\phi:X\rightarrow Y$,$\psi:Y\rightarrow X$,有复合:$\phi\circ 1_{X}=\phi$,$1_{X}\circ\psi=\psi$。
    \end{enumerate}
\end{enumerate}
\end{definition}
\begin{example}
定义线性空间范畴$\mathfrak{Lin}$(定义在复空间$\mathcal{C}$上),其中对象是线性空间,态射是线性空间之间的算子。这范畴有一个重要的满的子范畴:有限维空间范畴:$\mathfrak{FLin}$。
\end{example}
\section{范畴下的极限}
在范畴论中,一个更加抽象的极限定义更关键的把握了数学结构中的泛性质,比如其他结构:乘积,反极限。其对偶的概念:余极限推广很多数学结构,比如余乘积,直和,直极限。

\begin{definition}
\end{definition}

\end{document}
